\documentclass[10pt]{article}
\usepackage{amsmath}
\usepackage{amssymb}
\usepackage{amsthm,mathtools}
\usepackage{amscd}
\usepackage{amsfonts}
\usepackage{graphicx}
\usepackage{color}
\usepackage{amsfonts}
\usepackage{amsthm}
\usepackage{psfrag} 
\usepackage{graphicx,url,newverbs}
\usepackage{caption}
%\usepackage{subcaption}
% algorithm
\usepackage[ruled,vlined]{algorithm2e}
\usepackage{algorithmic}

\usepackage{amsfonts,verbatim}
\usepackage{amssymb}
\usepackage{amsmath}
\usepackage{subfigure}
\usepackage{psfrag,curves,algorithm2e}
\usepackage{enumitem}

\newcommand{\fig}[1]{\mbox{Figure{~#1}}}

\newcommand{\ubar}[1]{\text{\b{$#1$}}}


\newcommand{\bx}{{\mathbf{x}}}
\newcommand{\bmu}{{\boldsymbol{\mu}}}
\newcommand{\blambda}{{\boldsymbol{\lambda}}}
\newcommand{\bu}{\mathbf{u}}
\newcommand{\bphi}{\boldsymbol{\phi}}
\newcommand{\bpoly}{\mathbf{p}}
\newcommand{\half}{\frac{1}{2}}
% comments
\newcommand{\SL}[1]{{\color{cyan}{\bf SL:} #1}}
\newcommand{\pzc}[1]{{\color{blue}{pzc: }#1}}

\usepackage{amsthm}
\newtheorem{thm}{Theorem}
\newtheorem{rem}{Remark}
\newtheorem{cor}{Corollary}
\newtheorem{lem}{Lemma}[section]

% vertical equal
\usepackage{mathtools}
\newcommand{\verteq}{\rotatebox{90}{$\,=$}}
\newcommand{\verticalequal}[2]{\underset{\overset{\verteq}{#2}}{#1}}

\usepackage[top=2.8cm,bottom=2.8cm,left=3.5cm,right=3.5cm]{geometry}

\begin{document}
\title{Show $(Im(E),Im(H))^T$ is a fixed point of $\Pi$}
\maketitle
Suppose $(E,H)^T$ is the solutions of the non-dimensionalized equation
\begin{align}
&i\omega E = \nabla\times H-J,\\
&i\omega H = -\nabla\times E,
\end{align}
We will obtain 
\begin{align}
&-\omega Im(E) = \nabla\times Re(H)-Re(J), \\
& \omega Re(E) = \nabla\times Im(H)-Im(J), \\
&-\omega Im(H) = -\nabla\times Re(E), \\
&\omega  Re(H) = -\nabla\times Im(E). 
\end{align}
Utilizing this relation, one can verify that 
\begin{align}
\tilde{E}(t) = Im(E) \cos(\omega t) + Re(E) \sin(\omega t), \\
\tilde{H}(t) = Im(H) \cos (\omega t) + Re(H) \sin (\omega t).
\end{align}
is a solution of
\begin{align}
&\partial_t \tilde{E} =  \nabla\times \tilde{H}- \sin(\omega t)Re(J)-\cos(\omega t) Im(J),\\
&\partial_t \tilde{H} = -\nabla\times \tilde{E}.
\end{align}


Substitute $( \tilde{E}_0,\tilde{H}_0)^T=(Im(E),Im(H))^T$ and $( \tilde{E}(t),\tilde{H}(t) )^T$ into the definition of filtering operator $\Pi$, we have
\begin{align}
\Pi \left(\begin{matrix}
Im(E)\\
Im(H)
\end{matrix}\right)&=
\Pi \left(\begin{matrix}
\tilde{E}_0
\tilde{H}_0
\end{matrix}\right)
=  \frac{2}{T}\int_{0}^T 
(\cos(\omega t)-\frac{1}{4})
\left(\begin{matrix}
\tilde{E}(t)\\
\tilde{H}(t)
\end{matrix}\right)dt\notag\\
&=\frac{2}{T}\int_{0}^T 
(\cos(\omega t)-\frac{1}{4})
\left(\begin{matrix}
Im(E) \cos(\omega t) + Re(E) \sin(\omega t)\\
Im(H) \cos (\omega t) + Re(H) \sin (\omega t)
\end{matrix}\right)dt\notag\\
&=\left(\begin{matrix}
Im(E)\\
Im(H)
\end{matrix}\right).
\end{align}


\end{document}
